A few interesting questions are posed to the reader.

\begin{itemize}
    \item Are there conditions in the Trilinears of a Triangle Center so that its locus is an ellipse? Please refer to
    Table~\ref{tab:center-trilinears} for a few examples showing no apparent pattern.
    \item Can the degree of the locus of a Triangle Center or Derived Triangle vertex be predicted based on its Trilinears?
    \item Is there a Triangle Center such that its locus intersects a straight line more than 6 times?
    \item Certain Triangles Centers have non-convex loci (e.g., $X_{67}$ at $a/b=1.5$ \cite{dsr_locus_gallery_2019}). What determines non-convexity?
    \item What determines the number of self-intersections of a given locus?
    \item In the spirit of \cite{corentin19,olga14}, how would one determine via complex analytic geometry, that $X_6$ is a quartic?
    \item What is the non-elliptic locus described by the summits of equilaterals erected over each orbit side (used in the construction of the Outer Napoleon Triangle \cite{mw}).  \cite[pl\#13]{dsr_playlist_2020}. What kind of curve is it?
    \item Within $X_1$ and $X_{100}$ only $X_i$, $i=13{\ldots}18$ have irrational Trilinears. $X_j$, $j=359,\,360,\,364,\,365,\,367$ are irrational and have loci which numerically are non-elliptic. Can any irrational Center produce an ellipse?
    \item $X_6$ is the isogonal conjugate of $X_2$. Though the latter's locus is an ellipse, the former's is a quartic. In the case of the isogonal pair $X_3$ and $X_4$ both are ellipses. What is the connection with isogonal (and/or isotomic transformations) and ellipticity?
\end{itemize}

\subsection{Videos and Media}

The reader is encouraged to browse our companion paper \cite{reznik2020-ballet} where intriguing locus phenomena are investigated. Additionally, loci can be explored interactively with our browser-based  \href{https://dan-reznik.github.io/ellipse-mounted-loci-p5js/}{app} \cite{darlan2020-ellipse-mounted}.

Videos mentioned herein are on a \href{https://bit.ly/2REOigc}{playlist} \cite{dsr_playlist_2020}, with links provided on Table~\ref{tab:playlist}.

\begin{table}
\small
\begin{tabular}{|c|l|l|l|}
\hline
{id} & Title & Section & \texttt{youtu.be/<.>}\\
\hline
{01} &
{Locus of $X_1$ is an Ellipse} & \ref{sec:intro} & \href{https://youtu.be/BBsyM7RnswA}{\texttt{BsyM7RnswA}} \\
{02} &
{Locus of Intouchpoints is non-elliptic} & \ref{sec:intro}, \ref{app:early} & \href{https://youtu.be/9xU6T7hQMzs}{\texttt{9xU6T7hQMzs}}\\
{03} &
{$X_9$ stationary at EB center} & \ref{sec:intro} & \href{https://youtu.be/tMrBqfRBYik}{\texttt{tMrBqfRBYik}} \\
{04} &
{Stationary Excentral Cosine Circle}  &
\ref{sec:intro} & \href{https://youtu.be/ACinCf-D_Ok}{\texttt{ACinCf-D\_Ok}} \\
{05} &
{Loci for $X_1\ldots{X_5}$ are ellipses} &
\ref{app:early} & \href{https://youtu.be/sMcNzcYaqtg}{\texttt{sMcNzcYaqtg}} \\
{06} &
{Elliptic locus of Excenters similar to rotated $X_1$} &
\ref{app:early} & \href{https://youtu.be/Xxr1DUo19_w}{\texttt{Xxr1DUo19\_w}} \\
{07} &
{Loci of $X_{11}$, $X_{100}$ and Extouchpoints are the EB} &\ref{app:early} & \href{https://youtu.be/TXdg7tUl8lc}{\texttt{TXdg7tUl8lc}} \\
{08} &
{Family of Derived Triangles} &
\ref{sec:loci_geom} & \href{https://youtu.be/xyroRTEVNDc}{\texttt{xyroRTEVNDc}} \\
{09} &
{Loci of Vertices of Derived Triangles} &
 \ref{app:early},\ref{sec:loci_geom}& \href{https://youtu.be/OGvCQbYqJyI}{\texttt{OGvCQbYqJyI}} \\
{10} &
{Peter Moses' 29 Billiard Points} &
\ref{sec:loci_geom} & \href{https://youtu.be/JdcJt5PExsw}{\texttt{JdcJt5PExsw}}\\
{11} &
{Convex Comb.: $X_1$-Intouch and $X_2$-Midpoint} &
\ref{sec:triple-winding} & \href{https://youtu.be/3Gr3Nh5-jHs}{\texttt{3Gr3Nh5-jHs}}\\
{12} &
{Convex Comb.: $X_3$-Midpoint and $X_4$-Altfoot} &
\ref{sec:triple-winding} & \href{https://youtu.be/HZFjkWD_CnE}{\texttt{HZFjkWD\_CnE}}\\
{13} &
{Oval Locus of the Outer Napoleon Summits} &
\ref{sec:conclusion} & \href{https://youtu.be/70-E-NZrNCQ}{\texttt{70-E-NZrNCQ}} \\
\hline
\end{tabular}
\caption{Videos mentioned in the paper available in a Youtube playlist \cite{dsr_playlist_2020}. The last column contains a clickable YouTube code.}
\label{tab:playlist}
\end{table}

