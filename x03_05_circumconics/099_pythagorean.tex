We have seen \cite{ronaldo2020-loci} that with $a/b>a_4=\sqrt{2\,\sqrt{2}-1}\simeq{1.352}$ the orbit family contains obtuse triangles amongst which there always are 4 right triangles (identical up to rotation and reflection). An interesting question is which EB aspect ratios produce primitive {\em Pythagorean} triangles, i.e., right triangles with sides $s_1,s_2,s_3$ for which $s_1,s_2$ are co-prime, and $s_3^2=s_1^2+s_2^2$.

Start with an elementary Pythagorean triangle $P_1=(0,0)$, $P_2=(s_1,0)$, and $P_3=(s_1,s_2)$ choosing $s_1,s_2$ integers such that $s_3=s_3=\sqrt{s_1^2+s_2^2}$ is an integer.
The Circumbilliard is given by
\[
E_9(x,y)= s_2 x^2+ \left( s_3-s_1-s_2 \right) xy+s_1{y}^{2}-s_1s_2x-s_1 \left( s_1-s_3 \right) y=0.\]

Computing the axes of its Circumbilliard we obtain: 

\begin{equation}
    [a/b](s_1,s_2,s_3)= \frac {s_1+s_2+\sqrt {s_3\left(3\,s_3-2\,s_1-2\,s_2\right) }}{\sqrt { \left( 
s_1+s_2+3\,s_3 \right)  \left( s_1+s_2-s_3 \right) }}
\end{equation}

%\begin{equation*}
  %  \lim_{s_2\to \infty} %r(s_2,s_2,\sqrt{2}s_2)=\sqrt{2\sqrt{2}-1}\simeq %1.352
%\end{equation*}
%The axis of the Circumbilliard are:

%\begin{align}
%    a(m,n,p)=&  \\
 %   b(m,n,p)=&
%\end{align}

Table~\ref{tab:pythagorean} shows the aspect ratios required for the first 16 Pythagorean triples ordered by hypotenuse\footnote{The $a/b$ which produces $3:4:5$ was first computed in \cite{ronaldo2020-loci} in connection with $X_{88}$.}

\begin{table}[H]
\scriptsize
$$
\begin{array}{r|c|l|l}
(s_1,s_2,s_3) & a/b & {\simeq}a/b & \text{isosc. rank} \\
\hline
3, 4, 5 & {(7+\sqrt{5})\sqrt{11}}/{22} &  1.392 &  4 \\
5, 12, 13 & {\sqrt{14} (\sqrt{65}+17)}/{56} & 1.674 &  10 \\
8, 15, 17 & {\sqrt{111} (\sqrt{85}+23)}/{222} & 1.529 & 8\\
7, 24, 25  & {\sqrt{159} (5\,\sqrt{13}+31)}/{318} & 1.944 & 12 \\
\mathbf{20, 21, 29} & \mathbf{\sqrt{6} (\sqrt{145}+41)/{96}} & \mathbf{1.353} &  \mathbf{1} \\
12, 35, 37 & {\sqrt{395} (\sqrt{629}+47)}/{790} & 1.813 & 11 \\
9, 40, 41 & {\sqrt{86} (5\,\sqrt{41}+49)}/{344}& 2.184 &  14 \\
28, 45, 53 & {\sqrt{290} (\sqrt{689}+73)}/{1160} & 1.457 & 5 \\
11, 60, 61 & {\sqrt{635} (\sqrt{2501}+71)}/{1270} & 2.401 & 15 \\
16, 63, 65 & {\sqrt{959} (\sqrt{2405}+79)}/{1918} & 2.067 & 13 \\
33, 56, 65 & {\sqrt{426} (\sqrt{1105}+89)}/{1704} & 1.481  & 6 \\
48, 55, 73 & {\sqrt{2415} (\sqrt{949}+103)}/{4830} & 1.361 &  3 \\
\mathbf{13, 84, 85} & \mathbf{\sqrt{66} (97+\sqrt{5185})/{528}} & \mathbf{2.600} & \mathbf{16} \\
36, 77, 85 & {\sqrt{161} (\sqrt{2465}+113)}/{1288} & 1.602 & 9 \\
39, 80, 89 & {\sqrt{2895} (\sqrt{2581}+119)}/{5790} & 1.578 &  8 \\
65, 72, 97 & {\sqrt{1070} (\sqrt{1649}+137)}/{4280}& 1.357 & 2
\end{array}
$$
%\[a/b_{\perp}(m,n,p)=\sqrt {{\frac {m+n+\sqrt {p %( -2\,m-2\,n+3\,p \right) }}{m+n-
%\sqrt {p ( -2\,m-2\,n+3\,p \right) }}}}\]
\caption{The first column shows the first 16 Pythagorean triples ordered by hypotenuse $s_3$. The second one shows the $a/b$ family which produces it. The boldfaced triple $(20,21,29)$ (resp.~$(13,84,85)$) is the closest (resp.~farthest) to an isosceles in the table. This means $a/b$ will be closest to $a_4=1.352$. If rows were listed in ascending order of  $|s_1/s_2-1|$ (how scalene a triple is), the values of $a/b$ would also appear ascending, as indicated by the ``rank'' column.}
\label{tab:pythagorean}
\end{table}
%


\subsection{First 12k Pythagoreans in $a,b$ space}

Euclid's Algorithm generates a primitive Pythagorean triple from a co-prime pair $m,n$, $m>n$. For example, $(3,4,5)$ (resp.~$(5,12,13)$) is generated with $m=2,n=1$ (resp., $m=3,n=2$), etc.

Figure~\ref{fig:pythagorean} shows the first 12231 triangles generated in this fashion with $1<m<200$ in both $(s_1,s_2)$ and $(a,b)$ space. A few interesting observations are given in the caption. 

\begin{figure}[H]
    \centering
    \includegraphics[width=\textwidth]{pics_eps/0101_pythagorean_legs_ab.eps}
    \caption{\textbf{Top}: Right-triangles generated by Euclid's Algorithm with $n<m<200$, $m,n$ co-prime. Roughly 12k $(s_1,s_2)$ leg combinations are shown as black dots. To save space, the y-axis is drawn half-scale and points with $s_2>50000$ are not drawn. The green line represents $s_1=s_2$, i.e., isosceles triangles (only quasi-isosceles can be generated). The red (resp.~blue) curve is drawn over triangles obtained with $m=150$ (resp.~$n=150$). \textbf{Bottom}: Axes $(a,b)$ combinations which can produce the aforementioned right-triangles. The point cloud is properly under the green line $a/b=a_4$ (aspect ratio which produces an isosceles right-triangle). The constant $m=150$ (resp.~$n=150$) runs produce a red (resp.~blue) arch in $(a,b)$ space. Quasi-similar triangles organize in straight lines from the origin toward the right, i.e., they are generated by quasi-similar billiards.}
    \label{fig:pythagorean}
\end{figure}

\subsection{Iso-perimetric Primitive Pythagoreans}

Out of the 12231 Pythagorean triples generated, only 11548 perimeters $s_1+s_2+s_3$ are unique. Out of these, 10912 are associated with a single triple. An additional 585 (resp.~47) unique perimeters are each produced by two (resp.~3) distinct triples. Finally, 2 unique perimeters -- 60060 and 78540 -- are shared by two distinct groups of four triples, listed below:

$$
\scriptsize
\begin{array}{|c|c|c|c|}
\hline
\text{L} & s_1 & s_2 & s_3 \\
\hline
 60060 & 6699 & 26260 & 27101 \\
 60060 & 15960 & 19162 & 24938 \\
 60060 & 22035 & 12628 & 25397 \\
 60060 & 26936 & 5610 & 27514 \\
 78540 & 13515 & 31108 & 33917 \\
 78540 & 21896 & 24090 & 32554 \\
 78540 & 25179 & 20740 & 32621 \\
 78540 & 34440 & 8602 & 35498 \\
 \hline
\end{array}
$$

The perimeter $L$ of a 3-periodic orbit as a function of the EB axes $a,b$ is given by \cite{ronaldo2020-loci}:

\begin{equation*}
L=\frac{2(\delta+a^2+b^2)\sqrt{2\delta-a^2-b^2}}{a^2-b^2}
\end{equation*}

\noindent with $\delta=\sqrt(a^4-a^2 b^2+b^4)$. Each $L$ defines a level curve in $(a,b)$ space\footnote{Given by the quartic $(a^2-b^2)^2 L^4-(2(2a^2-b^2))(a^2-2 b^2)(a^2+b^2) L^2-27 a^4 b^4=0$.} which represents a 1d family of EBs such that all their orbit families have the same perimeter, see Figure~\ref{fig:iso-perimeter-3-and-4}.

The two groups of 4 triples with the same perimeter are shown in Figure~\ref{fig:iso-perimeter-4} against level curves of their perimeter. A similar picture for the 4-groups and 3-groups is shown in Figure~\ref{fig:iso-perimeter-3-and-4}.

\begin{figure}[H]
    \centering
    \includegraphics[width=.66\textwidth]{pics_eps/0120_fourTriplesPlot.eps}
    \caption{Two iso-perimetric curves are shown in $(a,b)$ space corresponding to EB families whose orbits have the same perimeter. Also shown are the 4-member groups of primitive Pythagorean triples sharing either perimeter.}
    \label{fig:iso-perimeter-4}
\end{figure}

\begin{figure}[H]
    \centering
    \includegraphics[width=.66\textwidth]{pics_eps/0121_fourTriplesAllPlot}
    \caption{All 4- and 3-groups of Pythagorean triples with the same perimeter. They fall on common iso-perimeter level curves (red) in $(a,b)$ space.}
    \label{fig:iso-perimeter-3-and-4}
\end{figure}