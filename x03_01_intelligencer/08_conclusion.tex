
The reader is invited to check out more details about this work on a companion \href{https://dan-reznik.github.io/Elliptical-Billiards-Triangular-Orbits/}{website} \cite{reznik_web}. We part with the following questions:

\begin{itemize}
    \item For $N=3$, what determines whether a triangle center or derived vertex produces an elliptic vs a more complicated locus? Which loci are algebraic, and which are not? Is there a general theory?
    \item Are there ellipsoidal (3d) counterparts to these invariants?
    \item Which invariants are still true for self-intersecting orbits? (consider videos for \href{https://youtu.be/cCYxN7ueGV4}{N=4} and \href{https://youtu.be/ECe4DptduJY}{N=5} \cite[pl\#3,4]{dsr_math_intell_playlist})
    \item Are there invariants for non-billiard (Poncelet) orbit families, e.g., created with two non-confocal or misaligned ellipse pairs, as shown in \cite{sergei2016}? A recent \href{https://youtu.be/B5dRXT8Xerw}{Video} \cite[pl\#22]{dsr_math_intell_playlist} shows certain aligned Poncelet pairs can collapse the locus of Triangular Centers to a point ($X_1$,$X_2$,$X_3$,$X_4$,$X_6$). Can an explicit map be found where stationary points under some special Poncelet Pair are identified with the Mittenpunkt of a Billiard (confocal) pair?
    \item Are there properties of triangle centers if orbits are defined on the surface of a sphere where edges become arcs of great circles (geodesics)? How about on the surface of an ellipsoid, bounded by the lines of curvature? Such Billiards are also known to be integrable \cite{sergei2002}.
    \item In the spirit of Theorem~\ref{thm:cosine-sum}, can expressions be derived in terms of $\gamma,L,N$ for the constant product of tangential polygon cosines and tangential-to-orbit area ratios? 
\end{itemize}

\subsection{List of Videos}
\label{sec:list-videos}

Videos mentioned above have been placed on a Youtube \href{https://bit.ly/2kTvPPr}{playlist} \cite{dsr_math_intell_playlist}.  Table~\ref{tab:playlist} contains quick-reference links to all videos mentioned, with column ``PL\#'' providing video number within the playlist.

\begin{table}[H]
\begin{tabular}{llll}
Title & Fig. & \href{https://bit.ly/2kTvPPr}{Links} & PL\#\\
\hline
{Open trajectories: varying input angle} &
\ref{fig:billiard-trajectories} &
\href{https://youtu.be/A7mPzrNJHkA}{N=3} &
{1} \\ 

{Locus of Incenter and Intouchpoints} &
\ref{fig:intro-plot},\ref{fig:locus-incenter} &
\href{https://www.youtube.com/watch?v=9xU6T7hQMzs}{N=3} &
{2}\\

{Self-intersecting orbits} &
-- &
\href{https://youtu.be/cCYxN7ueGV4}{N=4}, \href{https://youtu.be/ECe4DptduJY}{N=5} &
{3,4} \\

{Monge's Orthoptic Circle} &
\ref{fig:monge-orthoptic} & 
\href{https://youtu.be/9fI3iM2jrmI}{N=4} &
{5} \\

{Orbits and their Caustics} &
\ref{fig:6} &
\href{https://youtu.be/Y3q35DObfZU}{N=3\ldots{6}}& 
{6} \\

{Elliptic loci for $X_1\ldots{X_5}$} &
\ref{fig:7} & 
\href{https://youtu.be/sMcNzcYaqtg}{N=3} &
{7} \\

{Elliptic locus of the Excenters} &
\ref{fig:locus-incenter-excenter} &
\href{https://youtu.be/Xxr1DUo19_w}{N=3} &
{8} \\

{Loci of Medial, Intouch, Feuerbach Tri's} &
\ref{fig:non-elliptic} &
\href{https://youtu.be/OGvCQbYqJyI}{N=3} &
{9} \\

{Locus of Feuerbach Pt. and Anticompl.} &
\ref{fig:feuer_loci} &
\href{https://youtu.be/TXdg7tUl8lc}{N=3} &
{10} \\
 
{Locus of Orthic Incenter is 4-piece ellipse} &
-- &
\href{https://youtu.be/3qJnwpFkUFQ}{N=3} &
{11} \\

{Anticompl. Intouchpts. are on the Billiard} &
-- &
\href{https://youtu.be/50dyxWJhfN4}{N=3} & 
{12} \\

{Mittenpunkt stationary at Billiard center} & \ref{fig:mitten} &
\href{https://youtu.be/tMrBqfRBYik}{N=3} &
{13} \\

{The Circumbilliard} &
-- &
\href{https://youtu.be/vSCnorIJ2X8}{N=3} &
{14} \\

{Constant cosine sum and product} &
\ref{fig:conserve_cosines} &
\href{https://youtu.be/P8ykpE_ZbZ8}{N=3} &
{15} \\

{Stationary Cosine Circle} &
\ref{fig:cosine_circle_locus} &
\href{https://youtu.be/CrOSI8d8qDc}{v1},{ }
\href{https://youtu.be/hCQIT6_XhaQ}{v2} & 
{16,17} \\

{Generalized Mittenpunkt and Extouchpts.} &
\ref{fig:gen-mitten} &
\href{https://youtu.be/Bpc-MrR2IMc}{N=4,5},{ }
\href{https://youtu.be/TV2p7fPlYfE}{N{\textgreater}3} &
{18,19} \\

{Generalized stationary circle} &
\ref{fig:gen-circ-grid} &
\href{https://youtu.be/dINE4aH1cvk}{N=5},{ } \href{https://youtu.be/EFeINGIDFrg}{N=3\ldots{8}} &
{20,21} \\

{Loci with Poncelet Ellipse Pairs} &
-- &
\href{https://youtu.be/B5dRXT8Xerw}{N=3} &
{22}
\end{tabular}
\caption{Quick-reference to videos mentioned in the paper. Column ``PL\#'' is the entry within the Youtube playlist \cite{dsr_math_intell_playlist}.}
\label{tab:playlist}
\end{table}

\noindent \textbf{Postscript}: Elegant proofs for some of the $N>3$ invariants have recently appeared \cite{akopyan2020-invariants,bialy2020-invariants}.
