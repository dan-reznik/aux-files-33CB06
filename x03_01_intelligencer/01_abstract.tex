\begin{abstract}
Can any secrets still be shed by that much studied, uniquely integrable, Elliptic Billiard? Starting by examining the family of 3-periodic trajectories and the loci of their Triangular Centers, one obtains a beautiful and variegated gallery of curves: ellipses, quartics, sextics, circles, and even a stationary point. Secondly, one notices this family conserves an intriguing ratio: Inradius-to-Circumradius. In turn this implies three conservation corollaries: (i) the sum of bounce angle cosines, (ii) the product of excentral cosines, and (iii) the ratio of excentral-to-orbit areas. Monge's Orthoptic Circle's close relation to 4-periodic Billiard trajectories is well-known. Its geometry provided clues with which to generalize 3-periodic invariants to trajectories of an arbitrary number of edges. This was quite unexpected. Indeed, the Elliptic Billiard did surprise us!

\keywords{elliptic billiard, periodic trajectories, integrability, triangle center, locus, loci, conservation, invariance, invariant, constant of motion}

\noindent \textbf{MSC} {37-40, 51N20, 51M04, 51-04}
\end{abstract}