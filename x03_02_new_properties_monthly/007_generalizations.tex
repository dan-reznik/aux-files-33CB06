Theorem~\ref{thm:rovR-explicit} motivated us to conjecture that for all $N$

\begin{equation*}
\sum_{i=1}^{N}{\cos\theta_i}=1+\frac{r}{R}=\gamma L - N\cdot
\end{equation*}

\noindent To our delight, this was proved \cite{akopyan2020-invariants,bialy2020-invariants}. Likewise, observing that the excentral-to-orbit area ratio $A'/A$ is invariant for $N=3$ (see equation \eqref{eqn:area-ratio}) motivates querying via experiment whether this quantity remains numerically invariant for $N>3$. Indeed, one observes it does, however only for {\em odd} N. This was also subsequently proved \cite{akopyan2020-invariants}.

Recently, we've numerically detected dozens of new invariants for $N$-periodic orbits \cite{reznik2020-forty-invariants}, many of which have already been   established rigorously. Indeed, curiosity-driven experimentation with the elliptic billiard and related Poncelet families has proven an effective and entertaining approach with which to discover new properties and/or invariants.
